\documentclass[12pt]{article}
\usepackage[utf8]{inputenc}
\usepackage[T1]{fontenc}
\usepackage{pdflscape} 
\usepackage{lmodern}
\usepackage[a4paper,bindingoffset=0.2in,%
            left=0.5in,right=0.5in,top=0.5in,bottom=1in,%
            footskip=.25in]{geometry}
\usepackage[colorlinks=true, linkcolor=Black, urlcolor=Blue]{hyperref}
\usepackage{graphicx}
\usepackage{subcaption}
\usepackage{listings}
\usepackage{color}
\usepackage[table]{xcolor}
\definecolor{lightgray}{gray}{0.9}

\definecolor{codegreen}{rgb}{0,0.6,0}
\definecolor{codegray}{rgb}{0.5,0.5,0.5}
\definecolor{codepurple}{rgb}{0.58,0,0.82}
\definecolor{backcolour}{rgb}{0.95,0.95,0.92}

\lstdefinestyle{mystyle}{
	backgroundcolor=\color{backcolour},   
	commentstyle=\color{codegreen},
	keywordstyle=\color{magenta},
	numberstyle=\tiny\color{codegray},
	stringstyle=\color{codepurple},
	basicstyle=\ttfamily\footnotesize,
	breakatwhitespace=false,         
	breaklines=true,                 
	captionpos=b,                    
	keepspaces=true,                 
	numbers=left,                    
	numbersep=5pt,                  
	showspaces=false,                
	showstringspaces=false,
	showtabs=false,                  
	tabsize=2
}


\begin{document}
\title{Wahrheit und methode\\
\large Sebastian Michoń}
\date{\vspace{-10ex}}
\maketitle

\section {Ideas regarding way of the capitalist - production}
There are many things, that have to be done, and they can be done in any possible order. These are:
\begin{enumerate}
	\item Procrastinating, while contemplating prose: in other words, proof-reading, sanity-checking.
	\item Procrastinating, while contemplating visualizations - detecting problems, detecting imprecision and better ways to do stuff.
	\item Contemplating upon used mechanism - dynamic software engineering.
	\item Writing mathemagical poetry - articles.
	\item Writing post-mathemagical poetry - code, that is.
	\item Writing documentation, so that the sanity levels of this app will not reach abysses of Hades.
\end{enumerate}

\section {Ideas regarding way of the feather - writing}
\begin{enumerate}
	\item \textbf{Statement Comprehension has to be playful, algo description - exactly opposite, bridge between both shall be colorful visualization.} Visualization with SCompreh shouldn't contain neither correctness nor complexity proof - just an idea of algorithm. Most mathemagicians should be able to deduce, why is something so-and-so, if they are given method and basic outline of its work (except for refined tricks, i.e. suffix automata or NTT).
\end{enumerate}

\section {Ideas regarding way of the sword - implementation}
\begin{enumerate}
	\item \textbf{Isolate logic from presentation.} Parts of logic shall be changed only during the beginning - in state maker it shall only be read.
	\item In palingnesia (after constructor and BeginningExecutor), the visualization buttons shall be cast to this.buttons.x, where x represents variable presented by button - it reduces pain of writing and debugging StateMaker.
	\item One cannot change this ... except from ephemeral variables explicitly after leaving BeginningExecutor.
\end{enumerate}
\section {Inner sanctumm - Temp.js and its contents}


\subsection {On the Unmaker}
\subsubsection{On The Unmaker, I: elements}
\begin {enumerate}
	\item {[0, button, color1, color2]} - paints button from color1 to color2.
	\item {[1, button, innerHTML1, innerHTML2]} - changes innerHTML of a button from innerHTML1 to innerHTML2.
	\item {[2, list, element]} - appends element to a list.
	\item {[3, 'variable\_name', value1, value2]} - changes this.variable\_name from value1 to value2.
	\item {[5, fun, inv\_fun, args]} - execute fun on arguments args; in order to reverse changes, uses inv\_fun on same arguments.
\end{enumerate}

\subsubsection{On The Unmaker, II: Inner works}
\begin {enumerate}
	\item Unmaker reverts changes done by state maker, that ought to be reversed (examples of changes, that shall not be reversed are variables changed only once and used only since then or innerHTML that may be constant during whole execution).
	\item Unmaker can paint, pop from list, change value or innerHTML to previous value or execute certain function with given arguments. It executes in reversed order compared to StateMaker.
	\item StateMaker has array staat, in which changes to "this" that ought to be reversed and changes to buttons are kept in format described above. The values of "this" variables don't change until the end of a function.
\end {enumerate}

\subsubsection{On The Maker and Unmaker, III: Use}
\begin {enumerate}
	\item Unmaker actions depend on clicking "Previous" button, they're coded in Temp.js, so one does not have to do anything in code of an algorithm.
\end {enumerate}
\subsection {Ephemeral variables}
\begin{enumerate}
	\item Ephemeral variables - this.ephemeral. ..., have no memory - their value at the end of state is always equal to null. Also, in the beginning they are null -> thus, they don't need to participate in transformator.
	\item Their aim is to simplify access to variables (particularly staat and passer) within the StateMaker.
	\item At the beginning of StateMaker one has to write\\
	staat=this.ephemeral.staat\\
	passer=this.ephemeral.passer
\end{enumerate}


\section{Animations - states and/or description}
\subsection{Primitive Root}
\begin{enumerate}
	\item 
\end{enumerate}

\subsection{NTT/FFT animation: states}
\begin {enumerate}
	\item Writing indexes and values of both polynomials to multiply.
	\item Formulate either primitive root (NTT) or starting point (FFT)
	\item Write first two roots (1 and either function of proot - NTT or starting point - FFT)
	\item Construct butterfly
	\item Rewrite sequence according to butterfly indices
	\item First part of double - calculate the left part of next polynomial - \(A_{i,x,2k}\)
	\item Second part of double - calculate the right part of next polynomial - \(A_{i,x,2k+1}\)
	\item Repeated values of polynomials in points (roots of unity)
	\item Multiplication of polynomials
	\item Showing inverse roots
	\item Ending (After interpolating C(x))- multiply by inversion
\end {enumerate}

\section{Legitimate criticism}
\subsection{General}
\begin {enumerate}
	\item Add full list of concerns regarding content and its backend to this Sprawko -> 3 days
	\item Refactor - prever, finito -> 5 days
	\item Rethink colors:
	\begin{enumerate}
		\item Painter -> mechanism to write -> 1 day
		\item Showing past which lead to present -> 1 day (green from IEP?)
		\item One last gold -> Use gold only for ending -> 1 day
		\item Pair of intertwined numbers - recolor -> 0.5 day
	\end{enumerate}
	\item Add all Temp.js to this file -> 1 day
	\item Refactor dict in base - Its quality is dubious -> 2 days
	\item Rethink what to show, and what not to - use of NTT, BinSear etc.
	\item Buttons and SCompreh move with scroll -> 1 day
	\item Divide NT from system -> 0.5 day
	\item statics system -> 0.5 day
	\item Variable naming convention -> 2 days
	\item Refactor - division of logic and presentation - 6 days
	\item Add one function for double transformation of btn and systematic built-in mechanism for handling next move - 2 days
	\item returning 2 list - state and next from StateMaker, also NextState+BeginningExecutor - list - 2 days 
	\item Finishers to usability - 0.5 day
\end{enumerate}
\subsection{Specific}
\begin {enumerate}
	\item Proot - What the hell was I thinking when I wrote stateless finisher? -> 0.5 day
	\item Perm rep, no rep - problematic height - one during start, second during finish; also, too big start - reps. -> 0.5 day
\end{enumerate}

\section{Future}
\begin{enumerate}
	\item Batch one:
	\begin {enumerate}
		\item Add Mobius -> 5 days
		\item Add Totient genral -> 1 day
		\item Add IEP generalization -> 1 day (just formula)
		\item Add DLog -> 8 days
		\item Add Partitions -> 10 days
	\end{enumerate}

	\item Batch two:
	\begin {enumerate}
		\item Add los catalanos -> 5 days
		\item Add prime counting -> 7 days
	\end{enumerate}
\end{enumerate}
\end{document}
